\documentclass[norsk, 12pt]{beamer}
% Add handout to documentclass options if wanted

\usepackage[latin1]{inputenc}
\usepackage[T1]{fontenc}
\usepackage{babel,textcomp}

\setlength{\arrayrulewidth}{1.6pt}
\renewcommand{\arraystretch}{1.2}
\newlength{\Tcalc}

\usepackage[T1]{fontenc}
\usepackage{graphicx}
\usepackage{amsmath}
\usepackage{amssymb}
\usepackage{amsbsy}
\usepackage{amsfonts}
\usepackage{color}
\usepackage{xcolor}
\usepackage{epstopdf}
\usepackage{fancyvrb}
\usepackage{parskip}
\usepackage{url}
\usepackage{listings}

\definecolor{javared}{rgb}{0.6,0,0} % for strings
\definecolor{javagreen}{rgb}{0.25,0.5,0.35} % comments
\definecolor{javapurple}{rgb}{0.5,0,0.35} % keywords
\definecolor{javadocblue}{rgb}{0.25,0.35,0.75} % javadoc

\lstset{language=Matlab,
basicstyle=\ttfamily
keywordstyle=\color{javapurple},%\bfseries,
stringstyle=\color{javared},
commentstyle=\color{javagreen},
morecomment=[s][\color{javadocblue}]{/**}{*/},
morekeywords={super, with},
% numbers=left,
% numberstyle=\tiny\color{black},
stepnumber=2,
numbersep=10pt,
tabsize=2,
showspaces=false,
% captionpos=b,
showstringspaces=false,
frame=  false,
breaklines=true}

\setbeamertemplate{frametitle}
{\begin{centering}\smallskip
\insertframetitle\par
\smallskip\end{centering}}
\setbeamertemplate{itemize item}{$\bullet$}
\setbeamertemplate{navigation symbols}{}
\setbeamertemplate{footline}[text line]{%
\hfill\strut{%
\scriptsize\sf\color{black!60}%
\quad\insertframenumber
   }%
    \hfill
}

% Define some colors:
\definecolor{DarkFern}{HTML}{407428}
\definecolor{DarkCharcoal}{HTML}{4D4944}
\colorlet{Fern}{DarkFern!85!white}
\colorlet{Charcoal}{DarkCharcoal!85!white}
\colorlet{LightCharcoal}{Charcoal!50!white}
\colorlet{AlertColor}{orange!80!black}
\colorlet{DarkRed}{red!70!black}
\colorlet{LightBlue}{blue!70!white}
\colorlet{DarkBlue}{blue!70!black}
\colorlet{DarkGreen}{green!70!black}

% Use the colors:
\setbeamercolor{title}{fg=Fern}
\setbeamercolor{frametitle}{fg=Fern}
\setbeamercolor{normal text}{fg=Charcoal}
\setbeamercolor{block title}{fg=black,bg=Fern!25!white}
\setbeamercolor{block body}{fg=black,bg=Fern!25!white}
\setbeamercolor{alerted text}{fg=AlertColor}
\setbeamercolor{itemize item}{fg=Charcoal}

\newcommand{\frn}[1]{\textcolor{Fern}{#1}}
\newcommand{\alrt}{\color{AlertColor}}
\newcommand{\bt}[1]{\textbf{#1}}
\newcommand{\kommando}[1]{\textcolor{AlertColor}{\texttt{\textbackslash #1}}}
\newcommand{\ds}{\displaystyle}

\title{Oppsummering av uke 1}
\author{}
\institute{}
\date{}

\setbeamertemplate{frametitle}{\vspace{0.5cm} \insertframetitle} 

\begin{document}
\pagestyle{empty}

\begin{frame}
\maketitle
Se \texttt{folk.uio.no/jvbrink} for mer detaljer
\end{frame}

\begin{frame}
� programmere betyr � lage dataprogrammer. Dette gj�r vi ved � skrive {\alrt kode} i et bestemt {\alrt programmeringsspr�k}. Koden er instrukser til datamaskinen. 
\vspace{1cm}
F�r vi kan g� igang med � kode m� vi ofte bryte opp oppgaven vi skal l�se i sm� biter.
\end{frame}

\begin{frame}[fragile]
Datamaskinen husker ting i form av {\alrt variabler}, vi kan opprette variabler ved � gi dem {\alrt navn} og {\alrt innhold}. 
\vspace{1cm}
Python gir ogs� variabelen en {\alrt type}, som vi kan sjekke ved � bruke kommandoen \verb+type()+.
\end{frame}
	
\begin{frame}[fragile]
Vi kan lage {\alrt lister} ved hjelp av firkantparanteser: {\alrt \verb+[+}, {\alrt \verb+]+}. Lister er variabler som inneholder flere ting, de kan inneholde s� mange elementer vi vil, av alle slag.
\vspace{1.5cm}
Vi kan finne lengden p� en liste ved � bruke {\alrt \verb+len()+}. Vi kan ogs� legge til fler elementer med {\alrt \verb+my_list.append()+}. Enkelte elementer kan leses ut eller endres ved {\alrt indeksering}: \verb+my_list[2]+. Husk at indekseringen begynner p� 0. S� f�rste element er \verb+my_list[0]+, andre element er \verb+my_list[1]+ og s� videre.
\end{frame}
	
\begin{frame}[fragile]
For � skrive ut noe til terminalen kan vi bruke {\alrt \verb+print+-kommandoen}, vi kan skrive ut b�de variabler og {\alrt tekststrenger}.
\vspace{1.5cm}
Vi kan skrive ut flere ting etterhverandre hvis vi skiller dem med komma: \verb+print ting1, ting2+, eller vi kan skrive ut tekster som vi fyller inn med variabler: {\alrt \verb+print "Hei, jeg heter %s" % name+}.
\end{frame}
	
\begin{frame}[fragile]
Vi kan sp�rre brukeren et sp�rsm�l med kommandoen {\alrt \verb+raw_input()+}, i parantesen skriver du sp�rsm�let som en tekststreng.
\end{frame}
	
\begin{frame}[fragile]
Det er lett � gj�re feil i programmering, men det gj�r ingenting. N�r vi kj�rer et program med feil i, f�r vi en {\alrt feilmelding} som pr�ver � fortelle oss hva som har g�tt galt.
\end{frame}




\end{document}