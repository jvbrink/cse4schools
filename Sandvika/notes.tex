\documentclass[a4paper, 11pt, notitlepage, english]{article}

\usepackage{babel, textcomp}
\usepackage[utf8]{inputenc}
\usepackage[T1]{fontenc, url}
\usepackage{amsmath, amssymb}
\usepackage{amsbsy, amsfonts}
\usepackage{graphicx, color, xcolor}
\usepackage{framed, parskip, titling}
\usepackage{flafter, caption, multicol}
\usepackage{verbatim, listings}
\usepackage{shadow}
\usepackage{url}
\usepackage{framed}
\usepackage{fancyvrb}
\usepackage{titling}


%\DeclareCaptionLabelSeparator{colon}{. }
\renewcommand{\captionfont}{\sffamily}
\renewcommand{\captionlabelfont}{\bf\sffamily}
\setlength{\captionmargin}{40pt}

\setcounter{tocdepth}{2}

% \lstset{language=c++}
% \lstset{basicstyle=\footnotesize\sffamily,
%   numbers=left,                   % where to put the line-numbers
%   numberstyle=\tiny\color{gray},  % the style that is used for the line-numbers
%   stepnumber=2, }
% \lstset{backgroundcolor=\color{white}}
% \lstset{frame=single}
% \lstset{stringstyle=\sffamily}
% \lstset{keywordstyle=\color{red}\bfseries}
% \lstset{commentstyle=\color{blue}}
% \lstset{showspaces=false}
% \lstset{showstringspaces=false}
% \lstset{showtabs=false}
% \lstset{tabsize=2}
% \lstset{breaklines}

\definecolor{javared}{rgb}{0.6,0,0} % for strings
\definecolor{javagreen}{rgb}{0.25,0.5,0.35} % comments
\definecolor{javapurple}{rgb}{0.5,0,0.35} % keywords
\definecolor{javadocblue}{rgb}{0.25,0.35,0.75} % javadoc

\lstset{language=c++,
basicstyle=\ttfamily\footnotesize,
keywordstyle=\color{javapurple}\bfseries,
stringstyle=\color{javared},
commentstyle=\color{javagreen},
morecomment=[s][\color{javadocblue}]{/**}{*/},
numbers=left,
numberstyle=\tiny\color{black},
stepnumber=2,
numbersep=10pt,
tabsize=2,
showspaces=false,
showstringspaces=false,
frame= single,
breaklines=true}



\usepackage{geometry}
\geometry{headheight=0.01mm}
\geometry{top=24mm, bottom=29mm, left=39mm, right=39mm}

\renewcommand{\arraystretch}{2}
\setlength{\tabcolsep}{10pt}
% \makeatletter
% \renewcommand*\env@matrix[1][*\c@MaxMatrixCols c]{%
%   \hskip -\arraycolsep
%   \let\@ifnextchar\new@ifnextchar
%   \array{#1}}
% \makeatother

\newcommand{\dd}[1]{\ \text{d}#1}
\newcommand{\f}[2]{\frac{#1}{#2}} 
\newcommand{\beq}{\begin{equation}}
\newcommand{\eeq}{\end{equation}}
\newcommand{\bra}[1]{\langle #1|}
\newcommand{\ket}[1]{|#1 \rangle}
\newcommand{\braket}[2]{\langle #1 | #2 \rangle}
\newcommand{\av}[1]{\left| #1 \right|}
\newcommand{\op}[1]{\widehat{#1}}
\newcommand{\braopket}[3]{\langle #1 | \op{#2} | #3 \rangle}
\newcommand{\ketbra}[2]{\ket{#1}\bra{#2}}
\newcommand{\pp}[1]{\frac{\partial}{\partial #1}}
\newcommand{\ppn}[1]{\frac{\partial^2}{\partial #1^2}}
\newcommand{\up}{$\uparrow$}
\newcommand{\down}{$\downarrow$}
\newcommand{\bt}[1]{\boldsymbol{#1}}
\newcommand{\mat}[1]{\textsf{\textbf{#1}}}
\newcommand{\I}{\boldsymbol{\mathcal{I}}}
\renewcommand{\d}{{\rm d}}

\makeatletter
\renewcommand*\env@matrix[1][*\c@MaxMatrixCols c]{%
  \hskip -\arraycolsep
  \let\@ifnextchar\new@ifnextchar
  \array{#1}}
\makeatother

% \title{{\centerline{\Huge Pilotprosjekt ved Sandvika vgs}} {\centerline{\LARGE }}}
\title{Prosjekt ved Sandvika vgs: \\ Numeriske beregninger i Fysikk 1}
\author{Jonas van den Brink}
\setlength{\droptitle}{2.5cm}

\begin{document}
\maketitle

Dette kurset har blitt holdt ved Sandvika vgs ved to anledninger, først i våren 2014, og deretter i våren 2015. Første gang deltok ca 30 elever, og andre gang ca 15 elever. Kurset bli gitt over hhv 5 uker og 4 uker, med to timer undervisning hver uke etter skoletid. Elevene som deltok hadde for det meste R1, Fysikk 1 og IT1, men det var unntak til dette - det var blant annet noen elever fra første gang som ikke hadde noe Fysikk.

Elevene som deltok i prosjektet gjorde følgende:
\begin{itemize}
\item Utlede differensialligninger for fallskjermhopp og strikkhopp fra enkle fysiske lover, og drøfte gyldigheten av ligningene.
\item Løse disse differensialligningene numerisk, samt finne kreftene hopperen blir utsatt for over tid.
\item Diskutere tolkning av de numeriske løsningene ved å lage grafiske plot, og utskrift av terminalhastigheter og lignende.
\item Elever får en innføring i bruk av python til løsning av matematiske/fysiske problemstilling. En introduksjon til differensialligninger og integrasjon.
\end{itemize}

Mer konkret ble blant annet disse oppgavene utført
\begin{itemize}
 \item Finne hastighet for fallskjermhopper som funksjon av tid $v(t)$, se hvordan terminalhastighet endrer seg når fallskjerm utløses.
 \item Beregne terminalhastighet analytisk og sammenligne med numerisk resultat.
 \item Plotte g-krefter som funksjon av tid, justere parametere for å finne en realistisk max g.
 \item Finne hastighet og posisjon av strikkhopper som funksjon av tid $v(t)$, $x(t)$.
 \item Finne en passende fjærkonstant $k$ ved prøving og feiling, og ev analytisk regning for spesielt interesserte.
 \item Plotte g-krefter i et strikkhopp og sammenligne disse med fallskjermhopp.
\end{itemize}


Følgende læreplanmål møtes
\begin{itemize}
 \item beskrive banen til en partikkel ved hjelp av parameterframstilling, og bruke derivasjon og integralregning til å regne ut posisjon, fart og akselerasjon når en av de tre størrelsene er kjent
 \item bruke matematiske modeller som kilde for kvalitativ og kvantitativ informasjon, presentere resultater og vurdere gyldighetsområdet for modellene
\item planlegge og gjennomføre egne undersøkelser og foreta relevante forsøk innen de forskjellige hovedområdene i faget
\item samle inn og bearbeide data og presentere og vurdere resultater og konklusjoner av forsøk og undersøkelser, med og uten digitale verktøy
\item bruke simuleringsprogrammer til å vise fenomener og fysiske sammenhenger
\end{itemize}





\end{document}
