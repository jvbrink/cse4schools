\documentclass[english, 12pt]{beamer}
% Add handout to documentclass options if wanted

\usepackage[utf8]{inputenc}
\usepackage[T1]{fontenc}
\usepackage{babel,textcomp}


\usepackage{graphicx}
\usepackage{amsmath}
\usepackage{amssymb}
\usepackage{amsbsy}
\usepackage{amsfonts}
\usepackage{color}
\usepackage{xcolor}
\usepackage{epstopdf}
\usepackage{fancyvrb}
\usepackage{parskip}
\usepackage{url}
\usepackage{listings}

\DeclareMathAlphabet{\mathbfit}{OML}{cmm}{b}{it}

\definecolor{javared}{rgb}{0.6,0,0} % for strings
\definecolor{javagreen}{rgb}{0.25,0.5,0.35} % comments
\definecolor{javapurple}{rgb}{0.5,0,0.35} % keywords
\definecolor{javadocblue}{rgb}{0.25,0.35,0.75} % javadoc

\lstset{language=python,
basicstyle=\ttfamily\scriptsize,
keywordstyle=\color{javapurple},%\bfseries,
stringstyle=\color{javared},
commentstyle=\color{javagreen},
morecomment=[s][\color{javadocblue}]{/**}{*/},
% numbers=left,
% numberstyle=\tiny\color{black},
stepnumber=2,
numbersep=10pt,
tabsize=4,
showspaces=false,
captionpos=b,
showstringspaces=false,
frame= single,
breaklines=true}


\setlength{\arrayrulewidth}{1.6pt}
\renewcommand{\arraystretch}{1.2}
\newlength{\Tcalc}


\setbeamertemplate{frametitle}
{\begin{centering}\smallskip
\insertframetitle\par
\smallskip\end{centering}}
\setbeamertemplate{itemize item}{$\bullet$}
\setbeamertemplate{navigation symbols}{}
\setbeamertemplate{footline}[text line]{%
\hfill\strut{%
\scriptsize\sf\color{black!60}%
\quad\insertframenumber
   }%
    \hfill
}

% Define some colors:
\definecolor{DarkFern}{HTML}{407428}
\definecolor{DarkCharcoal}{HTML}{4D4944}
\colorlet{Fern}{DarkFern!85!white}
\colorlet{Charcoal}{DarkCharcoal!85!white}
\colorlet{LightCharcoal}{Charcoal!50!white}
\colorlet{AlertColor}{orange!70!black}
\colorlet{DarkRed}{red!70!black}
\colorlet{LightBlue}{blue!70!white}
\colorlet{DarkBlue}{blue!70!black}
\colorlet{DarkGreen}{green!70!black}

% Use the colors:
\setbeamercolor{title}{fg=Fern}
\setbeamercolor{frametitle}{fg=Fern}
\setbeamercolor{normal text}{fg=Charcoal}
\setbeamercolor{block title}{fg=black,bg=Fern!25!white}
\setbeamercolor{block body}{fg=black,bg=Fern!25!white}
\setbeamercolor{alerted text}{fg=AlertColor}
\setbeamercolor{itemize item}{fg=Charcoal}

\newcommand{\frn}[1]{\textcolor{Fern}{#1}}
\newcommand{\alrt}{\color{AlertColor}}
\newcommand{\bt}[1]{\textbf{#1}}
\newcommand{\kommando}[1]{\textcolor{AlertColor}{\texttt{\textbackslash #1}}}
\newcommand{\ds}{\displaystyle}

\renewcommand{\d}{\textrm{d}}

\title{Programmeringsprosjekt Sandvika \\ {\small En introduksjon til numeriske beregninger}}
\author{Jonas van den Brink \\ \texttt{j.v.brink@fys.uio.no}}
\institute{\alrt Simula Research Laboratory \\ Oslo, Norway}

\setbeamertemplate{frametitle}{\vspace{0.5cm} \insertframetitle} 

\begin{document}
\pagestyle{empty}


\begin{frame}[fragile]
\begin{center}
{\Huge \color{DarkFern} Eulers Metode}

\vspace{1cm}

En metode for å løse differensialligninger
\end{center}
\end{frame}

\begin{frame}
\frametitle{Målet er å finne hastighet og posisjon, $\alrt v(t)$, $\alrt x(t)$, som funksjoner av tid}

Hvis akselerasjonen er konstant, så er dette enkelt:
\begin{align*}
\alrt v(t) &\alrt= v_0 + at, \\
\alrt x(t) &\alrt= x_0 + v_0t + \frac{1}{2}at^2.
\end{align*}

Problemet er nå at akselerasjonen ikke alltid er konstant.
\end{frame}

\begin{frame}[fragile]
\begin{center}
{\Huge \color{DarkFern} Hovedidéen}

\vspace{1cm}

Akselerasjonen vil alltid være tilnærmet konstant \\ hvis vi bare ser på et kort nok tidsintervall
\end{center}
\end{frame}

\begin{frame}[fragile]
\frametitle{Vi finner bevegelsen ved å regne oss fremover i tid, et lite steg i tid av gangen}

Introduserer et lite tidssteg, $\alrt \Delta t$. 

\visible<2-> {
Akselerasjonen endrer seg gjennom bevegelsen, så vi kaller akselerasjonen ved start for $\alrt a_0$.
}

\visible<3-> {	
Hvis $\alrt \Delta t$ er liten nok får vi
\begin{align*}
\alrt v(\Delta t) &\alrt= v_0 + a_0 \cdot \Delta t, \\
\alrt x(\Delta t) &\alrt= x_0 + v_0\Delta t + \frac{1}{2}a_0 \cdot \Delta t^2.
\end{align*}
}
\end{frame}

\begin{frame}[fragile]
\frametitle{Vi finner bevegelsen ved å regne oss fremover i tid, et lite steg i tid av gangen}

Vi kan nå bruke $\alrt v(\Delta t)$ og $\alrt x(\Delta t)$ til å finne en ``ny'' akeselerasjon, siden $\alrt a(\Delta t) = F(\Delta x, \Delta v, \Delta t)/m$. 

\visible<2-> {
Vi kan bruke den nye akselerasjonen $\alrt a(\Delta t)$ til å regne oss enda et steg fremover i tid
\begin{align*}
\alrt v(2\Delta t) &\alrt= v(\Delta t) + a(\Delta t) \cdot \Delta t, \\
\alrt x(2\Delta t) &\alrt= x(\Delta t) + v(\Delta t)  + \frac{1}{2}a(\Delta t) \cdot \Delta t^2.
\end{align*}
}

\visible<3-> {
Og dette kan vi fortstette med så lenge vi vil.
}
\end{frame}

\begin{frame}[fragile]
For hvert steg vi tar, beveger vi oss $\alrt \Delta t$ frem i tid. Vi ser altså bare på tidspunktene
$$\alrt t = 0, \Delta t, 2\Delta t, 3\Delta t, \ldots$$

\visible<2-> {	
La oss navngi disse som følger
$$\alrt t_i \equiv i\Delta t$$
Da blir
\begin{align*}
\alrt t_0 &\alrt= 0, \\
\alrt t_1 &\alrt= \Delta t, \\
\alrt t_2 &\alrt= 2\Delta t, \\
& \, \ \alrt \vdots
\end{align*}
}
\end{frame}

\begin{frame}
Vi navngir $\alrt v$ og $\alrt x$ på samme måte
$$\alrt v_i \equiv v(t_i), \qquad \alrt x_i \equiv x(t_i).$$

\visible<2-> {	
Slik at
\begin{align*}
\alrt v_0 &\alrt= v(t_0) = v(0) \\
\alrt v_1 &\alrt= v(t_1) = v(\Delta t) \\
\alrt v_2 &\alrt= v(t_2) = v(2\Delta t) \\
&\alrt\quad\vdots
\end{align*}
\begin{align*}
\alrt x_0 &\alrt= x(t_0) = x(0) \\
\alrt x_1 &\alrt= x(t_1) = x(\Delta t) \\
\alrt x_2 &\alrt= x(t_2) = x(2\Delta t) \\
&\alrt\quad\vdots
\end{align*}
}
\end{frame}

\begin{frame}
Vi kan da skrive bevegelsesligningene som
\begin{align*}
\alrt v_{i+1} &\alrt  = v_i + a_i\cdot \Delta t, \\
\alrt x_{i+1} &\alrt = x_i + v_i\cdot \Delta t + \frac{1}{2}a_i\cdot \Delta t^2, 
\end{align*}

Der vi finner akselerasjonen $a_i$ fra $\alrt F(x_i, v_i, t_i)/m$.
\end{frame}


\begin{frame}
\begin{center}
\includegraphics[width=\textwidth]{eulersstory0}
\end{center}
\end{frame}

\begin{frame}
\begin{center}
\includegraphics[width=\textwidth]{eulersstory1}
\end{center}
\end{frame}

\begin{frame}
\begin{center}
\includegraphics[width=\textwidth]{eulersstory2}
\end{center}
\end{frame}

\begin{frame}
\begin{center}
\includegraphics[width=\textwidth]{eulersstory3}
\end{center}
\end{frame}

\begin{frame}
\begin{center}
\includegraphics[width=\textwidth]{eulersstory4}
\end{center}
\end{frame}

\begin{frame}
\begin{center}
\includegraphics[width=\textwidth]{eulersstoryN}
\end{center}
\end{frame}


\begin{frame}[fragile]
\frametitle{Algoritme for Eulers metode}
for $i=0,1,2,3,\ldots, N-1$:
\begin{enumerate}
	\item Bruk de forrige resultatene $x_i$ og $v_i$ for å regne ut akselerasjonen: $\alrt a_i = F(x_i, v_i, t_i)/m$.
	\item Regn ut den nye farten: $\alrt v_{i+1} = v_i + a_i\Delta t$.
	\item Regn ut den nye posisjonen: $\alrt x_{i+1} = x_i + v_i\Delta t + \frac{1}{2}a_i\Delta t^2$.
\end{enumerate}
\visible<2-> {
$$\Downarrow$$

\lstinputlisting{lsteuler.py}
$$\alrt t_i \Rightarrow \texttt{t[i]} \qquad  v_i \Rightarrow \texttt{v[i]} \qquad  r_i  \Rightarrow \texttt{r[i]}$$
}

\end{frame}


\end{document}