\documentclass[english, 12pt]{beamer}
% Add handout to documentclass options if wanted

\usepackage[utf8]{inputenc}
\usepackage[T1]{fontenc}
\usepackage{babel,textcomp}


\usepackage{graphicx}
\usepackage{amsmath}
\usepackage{amssymb}
\usepackage{amsbsy}
\usepackage{amsfonts}
\usepackage{color}
\usepackage{xcolor}
\usepackage{epstopdf}
\usepackage{fancyvrb}
\usepackage{parskip}
\usepackage{url}
\usepackage{listings}

\DeclareMathAlphabet{\mathbfit}{OML}{cmm}{b}{it}

\definecolor{javared}{rgb}{0.6,0,0} % for strings
\definecolor{javagreen}{rgb}{0.25,0.5,0.35} % comments
\definecolor{javapurple}{rgb}{0.5,0,0.35} % keywords
\definecolor{javadocblue}{rgb}{0.25,0.35,0.75} % javadoc

\lstset{language=python,
basicstyle=\ttfamily\scriptsize,
keywordstyle=\color{javapurple},%\bfseries,
stringstyle=\color{javared},
commentstyle=\color{javagreen},
morecomment=[s][\color{javadocblue}]{/**}{*/},
% numbers=left,
% numberstyle=\tiny\color{black},
stepnumber=2,
numbersep=10pt,
tabsize=4,
showspaces=false,
captionpos=b,
showstringspaces=false,
frame= single,
breaklines=true}


\setlength{\arrayrulewidth}{1.6pt}
\renewcommand{\arraystretch}{1.2}
\newlength{\Tcalc}


\setbeamertemplate{frametitle}
{\begin{centering}\smallskip
\insertframetitle\par
\smallskip\end{centering}}
\setbeamertemplate{itemize item}{$\bullet$}
\setbeamertemplate{navigation symbols}{}
\setbeamertemplate{footline}[text line]{%
\hfill\strut{%
\scriptsize\sf\color{black!60}%
\quad\insertframenumber
   }%
    \hfill
}

% Define some colors:
\definecolor{DarkFern}{HTML}{407428}
\definecolor{DarkCharcoal}{HTML}{4D4944}
\colorlet{Fern}{DarkFern!85!white}
\colorlet{Charcoal}{DarkCharcoal!85!white}
\colorlet{LightCharcoal}{Charcoal!50!white}
\colorlet{AlertColor}{orange!70!black}
\colorlet{DarkRed}{red!70!black}
\colorlet{LightBlue}{blue!70!white}
\colorlet{DarkBlue}{blue!70!black}
\colorlet{DarkGreen}{green!70!black}

% Use the colors:
\setbeamercolor{title}{fg=Fern}
\setbeamercolor{frametitle}{fg=Fern}
\setbeamercolor{normal text}{fg=Charcoal}
\setbeamercolor{block title}{fg=black,bg=Fern!25!white}
\setbeamercolor{block body}{fg=black,bg=Fern!25!white}
\setbeamercolor{alerted text}{fg=AlertColor}
\setbeamercolor{itemize item}{fg=Charcoal}

\newcommand{\frn}[1]{\textcolor{Fern}{#1}}
\newcommand{\alrt}{\color{AlertColor}}
\newcommand{\bt}[1]{\textbf{#1}}
\newcommand{\kommando}[1]{\textcolor{AlertColor}{\texttt{\textbackslash #1}}}
\newcommand{\ds}{\displaystyle}

\renewcommand{\d}{\textrm{d}}

\title{Programmeringsprosjekt Sandvika \\ {\small En introduksjon til numeriske beregninger}}
\author{Jonas van den Brink \\ \texttt{j.v.brink@fys.uio.no}}
\institute{\alrt Simula Research Laboratory \\ Oslo, Norway}

\setbeamertemplate{frametitle}{\vspace{0.5cm} \insertframetitle} 

\begin{document}
\pagestyle{empty}


\begin{frame}[fragile]
\begin{center}
{\Huge \color{DarkFern} Vi regner oss \\ stegvis frem i tid}
\end{center}
\end{frame}

\begin{frame}
\begin{center}
\includegraphics[width=\textwidth]{../fig/eulersstory0}
\end{center}
\end{frame}

\begin{frame}
\begin{center}
\includegraphics[width=\textwidth]{../fig/eulersstory1}
\end{center}
\end{frame}

\begin{frame}
\begin{center}
\includegraphics[width=\textwidth]{../fig/eulersstory2}
\end{center}
\end{frame}

\begin{frame}
\begin{center}
\includegraphics[width=\textwidth]{../fig/eulersstory3}
\end{center}
\end{frame}

\begin{frame}
\begin{center}
\includegraphics[width=\textwidth]{../fig/eulersstory4}
\end{center}
\end{frame}

\begin{frame}
\begin{center}
\includegraphics[width=\textwidth]{../fig/eulersstoryN}
\end{center}
\end{frame}


\begin{frame}[fragile]
\frametitle{Algoritme for Eulers metode}
for $i=0,1,2,3,\ldots, N-1$:
\begin{enumerate}
	\item Bruk de forrige resultatene $x_i$ og $v_i$ for å regne ut akselerasjonen: $\alrt a_i = F(x_i, v_i, t_i)/m$.
	\item Regn ut den nye farten: $\alrt v_{i+1} = v_i + a_i\Delta t$.
	\item Regn ut den nye posisjonen: $\alrt x_{i+1} = x_i + v_i\Delta t + \frac{1}{2}a_i\Delta t^2$.
\end{enumerate}
\visible<2-> {
$$\Downarrow$$

\lstinputlisting{lsteuler.py}
$$\alrt t_i \Rightarrow \texttt{t[i]} \qquad  v_i \Rightarrow \texttt{v[i]} \qquad  r_i  \Rightarrow \texttt{r[i]}$$
}
\end{frame}

\begin{frame}
\frametitle{Eksempel: Vertikalt kast}

Vi kaster en tennisball rett opp i lufta med en starthastighet på 10 m/s fra 1 m over bakken.

\begin{center}
\includegraphics[width=\textwidth]{../fig/tennisball}
\end{center}

\textbf{Oppgave:} Finn hastigheten og høyden over bakken som funksjoner av tid: $\alrt y(t)$, $\alrt v(t)$. Se bort ifra luftmotstand.
\end{frame}

\begin{frame}
\lstinputlisting{shell.py}
\end{frame}

\begin{frame}
\lstinputlisting{shell2.py}
\end{frame}


\end{document}